\documentclass[a4paper,11pt]{article}
\usepackage[italian]{babel}
\usepackage[T1]{fontenc}
\usepackage[utf8]{inputenc}
\usepackage{amsmath}
\usepackage{amsthm}
\usepackage[a4paper,top=1.5cm,bottom=2cm,left=1.5cm,right=1.5cm]{geometry}
\usepackage{amsfonts}
\usepackage{color}
\usepackage{bbm}
\usepackage{datetime}
\usepackage[colorlinks=true]{hyperref}
\usepackage{graphicx}
\usepackage{verbatim}
\usepackage{titlepic}
%\usepackage{physics}
\usepackage{subfigure}
\usepackage{float}

\newcommand{\abs}[1]{\lvert#1\rvert}

\begin{document}

\title{\sc Homework 3}
\author{\sc Alice Schirinà}
\maketitle

Consideriamo un sistema di molecole monoatomiche che interagiscono tramite il potenziale 
\begin{equation*}
U(r) = A \frac{\sigma e^{-r/\sigma}}{r}
\end{equation*}
quando $ r <r_c $ e $U(r) = 0$ per $ r >r_c $. Scegliamo il lato della scatola cubica $ L/\sigma $ in maniera tale che sia $\rho \sigma^3=0.5$ e utilizziamo unità ridotte,
\begin{table}[H]
	\centering
	\begin{tabular}{cl} 
		\hline
		$Unit\grave{a}\ usuali$ & $Unit\grave{a}\ ridotte$ \\
		\hline
		$r$	&	$r^* = r/\sigma$	\\
		$t$	&	$t^* = t/\sigma \sqrt{A/m}$	\\
		$v$	&	$v^* = v\sqrt{m/A}$	\\
		$E$	&	$E^* = E/A$	\\
		$p$	&	$p^* = p \sigma^3/A$	\\
		$T$	&	$T^* = k_B T/A$	\\\hline
	\end{tabular}
\end{table}
\medskip
\noindent \section*{Configurazione iniziale}
Generiamo una configurazione in cui le molecole sono distribuite casualmente nella scatola di lato $ L/\sigma $ e con velocità anch'esse casuali per cui usiamo il comando $RNG(-0.5,0.5)$. In particolare, vogliamo che sia $\sum v^* = 0$. In generale, questo non accade per le nostre velocità per cui definiamo la quantità
\begin{equation*}
P^* = \sum_{i=0}^N v_i^*
\end{equation*}
calcoliamo le quantità
\begin{eqnarray*}
v_i^*' &=& v_i* - \frac{P^*}{N}\\
k^* &=& \sum_i \frac{1}{2} (v_i^*')^2
\end{eqnarray*}
e infine riscaliamo le velocità come segue
\begin{equation*}
\hat{v}_i^* = \alpha v_i^*'
\end{equation*}
dove 
\begin{equation*}
\alpha = \sqrt{\frac{0.8N}{k^*}}
\end{equation*}
In questo modo otteniamo particelle disposte casualmente nella scatola con velocità tali che la somma totale sia nulla e con energia cinetica per particella $k^*=0.8$.
\end{document}