\documentclass[a4paper,11pt]{article}
\usepackage[italian]{babel}
\usepackage[T1]{fontenc}
\usepackage[utf8]{inputenc}
\usepackage{amsmath}
\usepackage{amsthm}
\usepackage[a4paper,top=1.5cm,bottom=2cm,left=1.5cm,right=1.5cm]{geometry}
\usepackage{amsfonts}
\usepackage{color}
\usepackage{bbm}
\usepackage{datetime}
\usepackage[colorlinks=true]{hyperref}
\usepackage{graphicx}
\usepackage{verbatim}
\usepackage{titlepic}
%\usepackage{physics}
\usepackage{subfigure}
\usepackage{float}

\newcommand{\abs}[1]{\lvert#1\rvert}

\begin{document}

\title{\sc Homework 3}
\author{\sc Alice Schirinà}
\maketitle

\noindent Consideriamo un sistema di molecole monoatomiche che interagiscono tramite il potenziale 
\begin{equation*}
U(r) = A \frac{\sigma e^{-r/\sigma}}{r}
\end{equation*}
quando $ r <r_c $ e $U(r) = 0$ per $ r >r_c $. Scegliamo il lato della scatola cubica $ L/\sigma $ in maniera tale che sia $\rho \sigma^3=0.5$ e utilizziamo unità ridotte, cioè
\begin{table}[H]
	\centering
	\begin{tabular}{l} 
		\hline
		 $Unit\grave{a}\ ridotte$ \\
		\hline
		$r^* = r/\sigma$	\\
		$t^* = t/\sigma \sqrt{A/m}$	\\
		$v^* = v\sqrt{m/A}$	\\
		$E^* = E/A$	\\
		$p^* = p \sigma^3/A$	\\
		$T^* = k_B T/A$	\\\hline
	\end{tabular}
\end{table}
\medskip
\noindent \section*{Configurazione iniziale}
Generiamo una configurazione in cui le molecole sono distribuite casualmente nella scatola di lato $ L/\sigma $ e con velocità anch'esse casuali per cui usiamo il comando $RNG(-0.5,0.5)$. In particolare, vogliamo che sia $\sum v^* = 0$. In generale, questo non accade per cui definiamo la quantità
\begin{equation*}
P^* = \sum_{i=0}^N v_i^*
\end{equation*}
e ridefiniamo velocità ed energia cinetica come segue
\begin{eqnarray*}
{v_i^*}' &=& v_i^* - \frac{P^*}{N}\\
K^* &=& \sum_i \frac{1}{2} ({v_i^*}')^2\\
\end{eqnarray*}
e infine scaliamo le velocità
\begin{equation*}
\hat{v}_i^* = \alpha {v_i^*}'
\end{equation*}
dove 
\begin{equation*}
\alpha = \sqrt{\frac{0.8N}{K^*}}
\end{equation*}
In questo modo otteniamo particelle disposte casualmente nella scatola con velocità tali che la somma totale sia nulla e con energia cinetica per particella $K^* = 0.8$.\\
Con questa configurazione eseguiamo una simulazione utilizzando lo schema di Verlet per le velocità con passo temporale $\Delta t^* = 0.002$ e fermandoci a $t^* = 1$. Infine, scaliamo ancora una volta le velocità in maniera tale che $K^*/N = 1$. Questa è la nostra configurazione iniziale.

\noindent Partendo dalla stessa configurazione eseguiamo sei simulazioni  ciascuna delle quali con passo tempo diverso
\begin{table}[H]
	\centering
	\begin{tabular}{cl} 
		\hline
		$run$	&	$\Delta t$ \\
		\hline
		$1$	&	$0.001$\\
		$2$	&	$0.003$	\\
		$3$	&	$0.009$	\\
		$4$	&	$0.027$	\\
		$5$	&	$0.081$	\\
		$6$	&	$0.243$	\\\hline
	\end{tabular}
\end{table}
\medskip
\noindent e arrestandoci a $t^* = 25$. Mostriamo di seguito gli andamenti dell'energia cinetica per molecola, $K^*/N$, per le varie simulazioni.
%grafici
Come possiamo osservare dal grafico, l'energia nella simulazione con $\Delta t = 0.243$ risulta essere instabile per cui eseguiamo la nostra analisi su tutte le simulazioni eccetto questa.

\section*{Divergenza della traiettoria}











 
\end{document}